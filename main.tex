\documentclass{beamer}

\usepackage[utf8]{inputenc}
\usepackage{hyperref}
\usepackage{lmodern}
\usetheme[style=light]{Nord}
\usefonttheme{professionalfonts}
\usepackage{amssymb, amsmath}
\usepackage [lambda,
advantage,
operators,
sets,
adversary,
landau,
probability,
notions,
logic,
ff,
mm,
primitives,
events,
complexity,
asymptotics,
keys]{ cryptocode }

\usepackage{amsfonts}

\usepackage[style=british]{csquotes}

\def\signed #1{{\leavevmode\unskip\nobreak\hfil\penalty50\hskip1em
  \hbox{}\nobreak\hfill #1%
  \parfillskip=0pt \finalhyphendemerits=0 \endgraf}}

\newsavebox\mybox
\newenvironment{aquote}[1]
  {\savebox\mybox{#1}\begin{quote}\openautoquote\hspace*{-.7ex}}
  {\unskip\closeautoquote\vspace*{1mm}\signed{\usebox\mybox}\end{quote}}

\newcommand{\overbar}[1]{\mkern 1.5mu\overline{\mkern-1.5mu#1\mkern-1.5mu}\mkern 1.5mu}


%Information to be included in the title page:
\title{Elliptic Curve Cryptography}
\subtitle{an introduction which is entirely too short}
\author{Giacomo Fenzi}
\institute{ETH Zurich}
\date{6 January 2022}



\begin{document}

\frame{\titlepage}

\begin{frame}
    \frametitle{Motivation}
    \begin{aquote}{Serge Lang}
    It is possible to write endlessly on elliptic curves. \\ (This is not a threat.)
    \end{aquote}

    \begin{itemize}
        \item Elliptic curves are everywhere in cryptography
        \item Coolest post quantum cryptography proposal
        \item Maths is banging
    \end{itemize}
\end{frame}

\begin{frame}
    \frametitle{Outline}
    \begin{itemize}
        \item Historical Notes
        \item Mathematical Background 
        \item Addition on Elliptic Curves
        \item Discrete Logarithm and Diffie Hellman
        \item Pairings
        \item Isogenies
    \end{itemize}
\end{frame}
\section{History}

\begin{frame}
    \frametitle{Diophantine Equations}
    Historically originated in the context of solving Diophantine equations such as
    \[ X^n + Y^n = Z^n, \; \; X,Y,Z \in \ZZ \]
    or equivalently
    \[ x^n + y^n = 1, \; \; x, y \in \QQ \]
    Often very hard, and in general undecidable\footnote{In fact, already undecidable with 11 integers variables!}! 

    Let us see what we can do...

\end{frame}

\begin{frame}
    \frametitle{One variable}
    \[ a_n x^n + a_{n-1} x^{n-1} + \dots a_1 x + a = 0 \]
    Quite easy! We can show that:
    \begin{theorem}
        Let $\frac{p}{q} \in \QQ$ be a solution of the above equation. 
        Then $q$ divides $a_n$ and $p$ divides $a_0$.
    \end{theorem}
    Check the finite list of candidates.

    Alternatively, solve numerically and find candidate of form $\frac{b}{a_n}$
\end{frame}

\begin{frame}
    \frametitle{Linear and Quadratic}
    \[ a x + b y = c \]
    \begin{theorem}
        Has infinitely many rational solution. If $\gcd(a, b)$ does not divide $c$, then no integers solutions.
        Else, infinitely many.
    \end{theorem}
    \[ a x^2 + b x y + c y^2  + d x + e y + f = 0 \]
    These are rational points on a conic. 
    \begin{itemize}
        \item Given a rational point, all of them can be found geometrically
        \item Hasse principle allows us to test if a rational point exists
    \end{itemize}

\end{frame}

\begin{frame}
    \frametitle{Cubics}
    What about:
    \[ a x^3 + b x^2 y + c x y^2 + d y^3 + e x^2 + f x y + g y^2 + h x + i y + j = 0 \; ? \]
    This is the general form of an elliptic curve!
    We have that 
    \begin{theorem}[Mordell]
        If the curve is non singular, and it has a rational point then the group of rational points is finitely generated
    \end{theorem}
    But no equivalent of Hasse principle!

    \begin{center}
        \textbf{Elliptic Curves $\neq$ Ellipse}
    \end{center}
\end{frame}

\section{Background}
\begin{frame}
    \frametitle{Fields}
    \begin{definition}
        A field $\FF$ is set together with two operations $+, \cdot$ such that 
        \begin{itemize}
            \item $\FF$ is an abelian group under $+$ with identity $0$
            \item $\FF - \{ 0 \}$ is an abelian group under multiplication with identity $1$.
            \item For every $a,b,c \in \FF$ we have that $a (b + c) = ab + ac$
            \item $0 \neq 1$
        \end{itemize}
    \end{definition}
    Informally, we can add, subtract, multiply and divide non zero elements.
\end{frame}

\begin{frame}
    \frametitle{Finite Fields}
    We are mostly interested in finite fields. We have that:
    \begin{theorem}
        For every prime $p$, and every $n \in \ZZ^+$ there is an unique field of size $p^n$, which we denote 
        by either $\mathbb{GF}(p^n)$ or $\FF_{p^n}$
    \end{theorem}

    If $n = 1$, then $\FF_{p} = \ZZ_p$, if not we can write them as
    \[ \FF_{p^n} = \frac{\FF_{p}[X]}{(f(x))} \]
    where $f(x)$ is an irreducible polynomial of degree $n$.
\end{frame}

\begin{frame}
    \frametitle{Characteristic}
    For any field, $\mathrm{char}(\FF)$ is the least integer\footnote{Or $\infty$ if no such integer exists} $\ell$ such that 
    \[ \underbrace{1 + \dots 1}_{\ell \text{ times}} = 0 \]
    We have that $\mathrm{char}(\FF_{p^n}) = p$. 
\end{frame}

\begin{frame}
    \frametitle{Field Extensions}
    Let $k, K$ be two fields. If there is an homomorphism $k \to K$, we can identify $k$ with a subfield of $K$. 
    In that case, $K$ is a \textbf{field extension} of $k$ which we denote by $k \subseteq K$. 

    Given any field $K$ we can construct the algebraic closure $\overbar{K}$ which is the smallest algebraically closed extension containing $K$.

    Some examples:
    \begin{itemize}
        \item $\QQ \subseteq \mathbb{R} \subseteq \mathbb{C}$
        \item $\FF_p \subseteq \FF_{p^2} \subseteq \FF_{p^3} \dots \subseteq \overline{\FF}_{p}$
    \end{itemize}
\end{frame}

\section{Elliptic Curves}
\subsection{Representation and Group Law}
\begin{frame}
    \frametitle{Weierstrass Form} 
    \begin{align*}
        a x^3 + b x^2 y + c x y^2 + d y^3 +& e x^2 + f x y + g y^2 + h x + i y + j = 0 \\
        &\big\downarrow \\
        y^2 + a xy + b y &= x^3 + c x^2 + d x + e \\
        &\bigg\downarrow \; \mathrm{char}(\FF) \neq 2, 3 \\
        y^2 = x^3 &+ a x + b
    \end{align*}
    Much easier to manage!
\end{frame}

\begin{frame}
    \frametitle{Elliptic Curves}
    \begin{definition}
        Let $\FF$ be a field. An elliptic curve $E$ defined over a field $\FF$ (denoted by $E/\FF$) is given by 
        \[ E: y^2 = x^3 + a x + b \]
        for $a, b \in \FF$. For any extension $\FF \subseteq \mathbb{E}$ we define
        \[ E(\mathbb{E}) = \set{(x, y) \in \mathbb{E} \times \mathbb{E} \mid y^2 = x^3 + a x + b } \cup \set{ \infty } \]
    \end{definition}
    Mathematicians are often interested with $E(\QQ) \subseteq E(\mathbb{R}) \subseteq E(\mathbb{C})$ but we mostly consider the finite case.
\end{frame}

\begin{frame}
    \frametitle{Some elliptic curves}
    \framesubtitle{(In $E(\mathbb{R})$ since they look better...)}
    TODO: One singular with cusp, one node and three non singular
\end{frame}

\begin{frame}
    \frametitle{Fundamental Quantities}
    \begin{definition}
        Let $E: y^2 = x^3 + a x + b$ be an elliptic curve.

        The \textbf{discriminant} of $E$ is 
            \[ \Delta = -16 (4 a^3 + 27 b^2) \] 
        A curve is \textbf{singular} if $\Delta = 0$.

        If $E$ is non-singular the $j$\textbf{-invariant} of $E$ is
            \[ j(E) = -1728 \frac{(4 A)^3}{\Delta} \]
    \end{definition}
    \begin{theorem}
        Let $E, E'$ be two elliptic curves over $K$. Then $E \cong E'$ if and only if $j(E) = j(E')$. 
    \end{theorem}
\end{frame}

\begin{frame}
    \frametitle{The Group Law} 
    TODO: Picture group law
\end{frame}

\begin{frame}
    \frametitle{The Group Law: Formulae}
    Let $E: y^2 = x^3 + a x + b$ be an elliptic curve. Let $P_i = (x_i, y_i) \in E(K)$. Define 
    \[ -P_0 = (x_0, -y_0 )\]
    Now, for $P_1 + P_2$: 
    \begin{itemize}
        \item If $x_1 = x_2$ and $y_1 = -y_2$, then $P_1 + P_2 = \infty$
        \item If $P_1 = \infty$ then $P_1 + P_2 = P_2$, and viceversa.
        \item Let $x_3 = \lambda^2 - x_1 - x_2$, $y_3 = \lambda(x_1 - x_3) - y_1$ where $\lambda$ is defined as:
        \[ \lambda = \begin{cases}
            \frac{y_2 - y_1}{x_2 - x_1}, \; x_1 \neq x_2 \\
            \frac{3 x_1^2 + a}{2y_1},\; \text{otherwise}
        \end{cases} \]
    \end{itemize}
    This makes $E$ into an abelian group with identity $\infty$ 
\end{frame}

\begin{frame}
    \frametitle{Scalar multiplication}
    For $n > 0, P \in E$ we write $[n]P = \underbrace{P + \dots + P}_{n \text{ times}}$. We then extend the notation by letting $[0]P = \infty$ and 
    $[-n]P = [n](-P)$.

    Note that we can compute $[n]P$ in $\Theta(\log n)$ group operations using square and multiply.

    For $m \in \ZZ$ we can define a map $[m]: E \to E$ accordingly, and write:
    \[ E[m] \coloneqq \ker[m] \]
    to be the $m$-\textbf{torsion subgroup} of $E$.
\end{frame}

\begin{frame}
    \frametitle{Number of Points on a curve}
    Heuristically, we expect $\approx q + 1$ points
    \begin{theorem}[Hasse]
        Let $E$ be an elliptic curve defined over $\FF_q$.
        \[ \left|\#E(\FF_q) - q - 1\right| \leq 2 \sqrt{q} \] 
    \end{theorem}
    Exact value can be efficiently found using Schoof's algorithm in $O((\log q)^8)$.
\end{frame}

\subsection{Discrete Log Crypto}

\begin{frame}
    \frametitle{Discrete Logarithm}
    Cryptography relies on hardness assumptions. 
    \begin{definition}
        Let $\mathrm{Gen}(\secparam)$ be a p.p.t. algorithm that returns a group description $\GG = (+, P, q)$, where $\GG = \langle P \rangle$ and $q = \#\GG$.
        For an attacker $\adv$, define 
        \[\advantage{dlp}{\adv} = \Pr\left[\adv\left(\secparam, \GG, [k]P \right) = k \; \middle\vert \; \begin{matrix}
            \GG \sample \mathrm{Gen}(\secparam) \\
            k \sample \ZZ_q
        \end{matrix}
        \right] \]
        We say that the \textbf{discrete logarithm assumption} hold with respect to $\mathrm{Gen}$ if, for every p.p.t. attacker $\adv$, $\advantage{dlp}{\adv}[(\cdot)]$ is negligible.
    \end{definition}
\end{frame}

\begin{frame}
    \frametitle{Related Assumptions}
    In practice, we make stronger assumptions, such as Computational Diffie Hellman and Decisional Diffie Hellman. 

    \begin{itemize}
        \item CHD: From $[x]P, [y]P$ compute $[xy]P$
        \item DDH: Distinguish $(P, [x]P, [y]P, [xy]P)$ from $(P, [x]P, [y]P, [z]P)$
    \end{itemize}
    In fact, pairings make DDH easy on elliptic curves! 
    \[ \mathrm{DDH} \leq_R \mathrm{CDH} \leq_R\footnote{In fact equivalent} \mathrm{DLP} \]

    \textbf{Representation matters}! $\ZZ_{p-1} \cong \ZZ^*_p$ as groups but the discrete logarithm is trivial in the former, assumed hard in the latter. 
\end{frame}

\begin{frame}
    \frametitle{Why elliptic curves?}
    \begin{center}
        \begin{tabular}{c c c c}
            Assumption & Group   & Best Algorithm & $\approx$ Complexity \\
            RSA        & $\ZZ_N$ & Number Field Sieve & $\mathrm{exp}(c ^3\sqrt{\log N})$ \\
            DLP        & $\FF^*_p$ & Number Field Sieve & $\mathrm{exp}(c ^3\sqrt{\log p})$ \\
            DLP        & $E(\FF_p)$ & Pollard Rho & $\sqrt{p}$ \\
        \end{tabular}
    \end{center}
    \textbf{Best known attacks against ECC are generic attacks}
    \begin{itemize}
        \item Shorter keysizes ($\approx 256$ vs\footnote{For 128 bits of security} 3072 bits)
        \item Faster computation\footnote{against other DLP schemes and private RSA ops}
    \end{itemize}
\end{frame}

\begin{frame}
    \frametitle{EC Diffie Hellman Key Exchange}
    Let $E$ be an elliptic curve over $\FF_q$. Let $p$ be a large prime dividing $\#E(\FF_q)$ and $P$ a point of order $p$. 
    \procedureblock{Diffie Hellman}{
        \textbf{Alice} \> \> \textbf{Bob} \\
        x \sample \ZZ_q \> \> y \sample \ZZ_q \\
        Q_A = [x]P \> \> Q_B = [y]P \\
        \> \xrightarrow{Q_A} \> \\
        \> \xleftarrow{Q_B} \>  \\
        K = [x]Q_B \> \> K = [y]Q_A
    }
    Correctness follows since:
    \[ K = [x]Q_B = [x][y]P = [xy]P = [y][x]P = [y]Q_A = K \]

\end{frame}

\begin{frame}
    \frametitle{Easy Elliptic Curves}
    \textbf{DLP is not equally hard on every curve}!
    \begin{itemize}
        \item Singular curves over $\FF_p$. Equivalent to DLP in\footnote{Or in some small extension} $\FF_p^*$ or $\FF_p^+$
        \item Curves and subgroups with small embedding degree. E.g. supersingular and anomalous curves
        \item Curves that admit pairings to small finite fields.
        \item Curves defined over $\FF_{p^k}$ for $k$ with small factors. GHS Method, Diem's Analysis.
    \end{itemize}
\end{frame}

\begin{frame}
    \frametitle{Pollard Rho}
    Collision search for $f: S \to S$. Let $x_0 \in S$, $x_n = f(x_{n-1})$.
    TODO: Insert image
    
    Expected $\sqrt{\pi \#S/2}$ calls to $f$, constant memory.
\end{frame}

\begin{frame}
    \frametitle{Pollard Rho}
    Let $G$ be a group of order $N$. We want to find $k$ s.t. $[k]P = Q$.
    Split $G = A \sqcup B \sqcup C$ with $\# A \approx \# B \approx \#C$.
    Define 
    \[ f(X) = \begin{cases}
        P + X, \; &X \in A \\
        [2]X, \; &X \in B \\
        Q + X, \; &X \in C
    \end{cases} \]
    Let $X_0 = \infty$, then $X_i = [\alpha_i] P + [\beta_i] Q$ and we can track $\alpha_i, \beta_i$. 
    A collision $X_j = X_{j+\ell}$ with $\gcd(\beta_{j+\ell} - \beta_j, N) = 1$ allows us to solve the DLP with
    \[ k \equiv \frac{\alpha_j - \alpha_{j + \ell}}{\beta_{j+\ell} - \beta_j} \pmod{N} \]

\end{frame}

\subsection{Pairings}
\begin{frame}
    \frametitle{Pairings}
    \begin{definition}
        Let $\GG, \GG_T$ be two groups. A \textbf{pairing} is a map $e: \GG \times \GG \to \GG_T$ that is:
        \begin{itemize}
            \item Non degenerate: 
            \[ e(S, T) = 1 \; \forall S \in \GG \implies T = 0_{\GG}  \]
            \item Bilinear:
            \[ e(S_1 + S_2, T) = e(S_1, T)e(S_2, T)\]
            \[ e(S, T_1 + T_2) = e(S, T_1)e(S_2, T_2)\]
            \item Alternating:
            \[ e(T, T) = 1 \]
            
        \end{itemize}
    \end{definition}
\end{frame}

\begin{frame}
    \frametitle{Weil Pairing}
    Every elliptic curve $E$ over $K$ admits an efficiently computable pairing
    \[ e_m : E[m] \times E[m] \to \mu_m \]
    where $\mu_m$ is the group of $m$-th root of unity. 

    In degenerate on cyclic subgroups of $E[m]$, so use modified Weil pairing
    \begin{align*}
        \langle \cdot , \cdot \rangle : E[m] &\times E[m] \to \mu_m \\
        \langle P, Q \rangle &= e_m(S, \phi(Q))
    \end{align*}
    For $\phi: E \to E$ a distorsion map\footnote{If it exists}

\end{frame}

\begin{frame}
    \frametitle{BLS Signatures}
    Let $\GG, \GG_T$ be cyclic groups of prime order $p$. Let $P$ be a generator of $\GG$, and $e$ a non degenerate pairing.
    Also, let $H: \bin^* \to \GG$
    \begin{center}
        
    \begin{pcvstack}
        \begin{pchstack}
        \procedure{$\mathrm{Gen}(\secparam)$}{
            x \sample \ZZ_p \\
            pk \coloneqq [x]P \\
            sk \coloneqq x  \\
            \pcreturn (pk, sk)
        }

        \procedure{$\mathrm{Sign}(sk, m)$}{
            Q \leftarrow H(m) \\
            \sigma \leftarrow [x]Q \\
            \pcreturn \sigma
        }
        \end{pchstack}


        \procedure[space=auto]{$\mathrm{Verify}(pk, m, \sigma)$}{
            \pcreturn e(\sigma, P) =_? e(H(m), [x]P)
        }
    \end{pcvstack}
    \end{center}
    Correctness by:
    \[ e(\sigma, P) = e([x]Q, P) = e(Q, P)^x = e(Q, [x]P) = e(H(m), [x]P) \]
 
\end{frame}

\begin{frame}
    \frametitle{Post Quantum}
    \begin{itemize}
        \item<1-> Discrete logarithms, RSA, and pairings broken by Shor's algorithm
        \item<2-> Can we recover?
        \item<3-> Yes, lattices, codes, multinear maps...
        \item<4-> \textbf{Isogenies!}
    \end{itemize}
\end{frame}

\begin{frame}
    \frametitle{Isogenies}
    ``Nice maps'' between elliptic curves.
    \begin{definition}
        Let $E_1, E_2$ be elliptic curves. An \textbf{isogeny} is a morphism
        \[ \phi : E_1 \to E_2 \]
        with $\phi(\infty) = \infty$.
        If $\phi(E_1) \neq \set{\infty}$, $E_1$ is \textbf{isogenous} to $E_2$.
    \end{definition}
    For example, the curves $y^2 = x^3 + x$ and $y^2 = x^3 - 3x + 3$ are isogenous over $\FF_{71}$ via the isogeny 
    \[ (x, y) \mapsto \left(\frac{x^3 - 4 x^2 + 30 x -12}{(x - 2)^2}, y\cdot\frac{x^3 - 6x^2 -14x + 35}{(x - 2)^3} \right)\]
\end{frame}

\begin{frame}
    \frametitle{Properties of isogenies}

    \begin{itemize}
        \item Each isogeny is also a group homomorphism
        \item The map $[m]: E \to E$ is an isogeny
        \item You can compose isogenies
        \item Each isogeny has a degree, and it is multiplicative $\deg(\phi \circ \psi) = \deg(\phi)\deg(\psi)$
        \item Each isogeny $\phi: E_1 \to E_2$ has a unique dual $\hat{\phi}: E_2 \to E_1$ such that 
        \[ \phi \circ \hat{\phi} = [\deg(\phi)] \]
        \item An isogeny between two Weierstrass curves has the form
        \[ (x, y) \mapsto \left(\frac{f}{h^2}(x), y\cdot \frac{g}{h^3}(x) \right) \]
    \end{itemize}
\end{frame}

\begin{frame}
    \frametitle{Separable and Inseparable Isogenies}
    \begin{definition}
        Let $E/k: y^2 = x^3 + ax + b$, with $\mathrm{char}(k) = p$. Define $E^{(p^r)}: y^2 = x^3 + a^{p^r} x + b^{p^r}$.
        The map:
        \[ \pi: E \to E^{(p^r)}, (x, y) \mapsto \left(x^{p^r}, y^{p^r} \right) \]
        is the $(p^r)$-\textbf{Frobenius isogeny}. Note if $k = \FF_{p^r}$ then $E^{(p^r)} = E$
    \end{definition}
    If an isogeny factors trough a Frobenius isogeny it is inseparable. We are mostly 
    concerned with the separable case.

\end{frame}

\begin{frame}
    \frametitle{Kernel and Velu}
    \begin{theorem}
       There is a one to one correspondence between finite subgroups of elliptic curves 
       and separable isogenies from that curve, up to post-compostion with isomorphisms
       \[ \text{kernels} \longleftrightarrow \text{isogenies} \]

    \end{theorem}
       Let $E/k$, with $k$ a finite field. For any subgroup $H \leq E$ we can find an 
       isogeny with kernel $H$ in $\Theta(\#H)$ using Velu's formulas. We denote the target of 
       that isogeny by $E/H$
    
\end{frame}

\begin{frame}
    \frametitle{Computing large degree isogenies}
    For crypto, we want to compute isogeny of large degree i.e. with large kernels.
    Velu's formulas are too slow.
\end{frame}

\begin{frame}
    \frametitle{Isogeny Problems}
    It is easy to find out if two curves are isogenous
    \begin{theorem}
        Two curves $E_1, E_2$ over a finite field $k$ are isogenous over $k$ if and only if $\#E_1(k) = \#E_2(k)$.
    \end{theorem}
   Finding the isogeny is dramatically harder 
\end{frame}

\section{Conclusion}

\begin{frame}
    \frametitle{Resources}
    \begin{itemize}
        \item  J.H. Silverman, The Arithmetic of Elliptic Curves
        \item  J.H. Silverman, J.T. Tate, Rational Points on Elliptic Curves 
        \item  D.A. Cox, Primes of the form $x^2 + n y^2$
        \item L. Panny, notes: \href{https://yx7.cc/docs/misc/isog_bristol_notes.pdf}{[intro]} \href{https://yx7.cc/docs/misc/isogprob_bristol_notes.pdf}{[isogenies problems]}
    \end{itemize}
\end{frame}


\end{document}