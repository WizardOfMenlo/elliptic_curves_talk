\documentclass{beamer}

\usepackage[utf8]{inputenc}
\usepackage{hyperref}
\usepackage{lmodern}
\usetheme[style=light]{Nord}
\usefonttheme{professionalfonts}
\usepackage{amssymb, amsmath}
\usepackage [lambda,
advantage,
operators,
sets,
adversary,
landau,
probability,
notions,
logic,
ff,
mm,
primitives,
events,
complexity,
asymptotics,
keys]{ cryptocode }

\usepackage{tikz}
\usepackage{amsfonts}
\usetikzlibrary{arrows}

\usepackage[style=british]{csquotes}

\def\signed #1{{\leavevmode\unskip\nobreak\hfil\penalty50\hskip1em
  \hbox{}\nobreak\hfill #1%
  \parfillskip=0pt \finalhyphendemerits=0 \endgraf}}

\newsavebox\mybox
\newenvironment{aquote}[1]
  {\savebox\mybox{#1}\begin{quote}\openautoquote\hspace*{-.7ex}}
  {\unskip\closeautoquote\vspace*{1mm}\signed{\usebox\mybox}\end{quote}}

\newcommand{\overbar}[1]{\mkern 1.5mu\overline{\mkern-1.5mu#1\mkern-1.5mu}\mkern 1.5mu}


%Information to be included in the title page:
\title{Elliptic Curve Cryptography}
\subtitle{an introduction which is entirely too short}
\author{Giacomo Fenzi}
\institute{ETH Zurich}
\date{6 January 2022}



\begin{document}

\frame{\titlepage}

\begin{frame}
    \frametitle{Motivation}
    \begin{aquote}{Serge Lang}
    It is possible to write endlessly on elliptic curves. \\ (This is not a threat.)
    \end{aquote}
    \begin{itemize}
        \item 
    \end{itemize}
\end{frame}

\begin{frame}
    \frametitle{Outline}
    \begin{itemize}
        \item Historical Notes
        \item Mathematical Background 
        \item Addition on Elliptic Curves
        \item Discrete Logarithm and Diffie Hellman
        \item Pairings
        \item Isogenies
    \end{itemize}
\end{frame}

\begin{frame}
    \frametitle{History}
    Historically originated in the context of solving Diophantine equations such as
    \[ X^n + Y^n = Z^n, \; \; X,Y,Z \in \ZZ \]
    or equivalently
    \[ x^n + y^n = 1, \; \; x, y \in \QQ \]
    Often very hard, and in general undecidable\footnote{In fact, already undecidable with 11 integers variables!}! 

    Let us see what we can do...

\end{frame}

\begin{frame}
    \frametitle{History: One variable}
    \[ a_n x^n + a_{n-1} x^{n-1} + \dots a_1 x + a = 0 \]
    Quite easy! We can show that:
    \begin{theorem}
        Let $\frac{p}{q} \in \QQ$ be a solution of the above equation. 
        Then $q$ divides $a_n$ and $p$ divides $a_0$.
    \end{theorem}
    Check the finite list of candidates.

    Alternatively, solve numerically and find candidate of form $\frac{b}{a_n}$
\end{frame}

\begin{frame}
    \frametitle{History: Linear and Quadratic}
    \[ a x + b y = c \]
    \begin{theorem}
        Has infinitely many rational solution. If $\gcd(a, b)$ does not divide $c$, then no integers solutions.
        Else, infinitely many.
    \end{theorem}
    \[ a x^2 + b x y + c y^2  + d x + e y + f = 0 \]
    These are rational points on a conic. 
    \begin{itemize}
        \item Given a rational point, all of them can be found geometrically
        \item Hasse principle allows us to test if a rational point exists
    \end{itemize}

\end{frame}

\begin{frame}
    \frametitle{History: Cubics}
    What about:
    \[ a x^3 + b x^2 y + c x y^2 + d y^3 + e x^2 + f x y + g y^2 + h x + i y + j = 0 \; ? \]
    This is the general form of an elliptic curve!
    We have that 
    \begin{theorem}[Mordell]
        If the curve is non singular, and it has a rational point then the group of rational points is finitely generated
    \end{theorem}
    But no equivalent of Hasse principle!

    \begin{center}
        \textbf{Elliptic Curves $\neq$ Ellipse}
    \end{center}
\end{frame}

\begin{frame}
    \frametitle{Background: Fields}
    \begin{definition}
        A field $\FF$ is set together with two operations $+, \cdot$ such that 
        \begin{itemize}
            \item $\FF$ is an abelian group under $+$ with identity $0$
            \item $\FF - \{ 0 \}$ is an abelian group under multiplication with identity $1$.
            \item For every $a,b,c \in \FF$ we have that $a (b + c) = ab + ac$
            \item $0 \neq 1$
        \end{itemize}
    \end{definition}
    Informally, we can add, subtract, multiply and divide non zero elements.
\end{frame}

\begin{frame}
    \frametitle{Background: Finite Fields}
    We are mostly interested in finite fields. We have that:
    \begin{theorem}
        For every prime $p$, and every $n \in \ZZ^+$ there is an unique field of size $p^n$, which we denote 
        by either $\mathbb{GF}(p^n)$ or $\FF_{p^n}$
    \end{theorem}

    If $n = 1$, then $\FF_{p} = \ZZ_p$, if not we can write them as
    \[ \FF_{p^n} = \frac{\FF_{p}[X]}{(f(x))} \]
    where $f(x)$ is an irreducible polynomial of degree $n$.
\end{frame}

\begin{frame}
    \frametitle{Background: Characteristic}
    For any field, $\mathrm{char}(\FF)$ is the least integer\footnote{Or $\infty$ if no such integer exists} $\ell$ such that 
    \[ \underbrace{1 + \dots 1}_{\ell \text{ times}} = 0 \]
    We have that $\mathrm{char}(\FF_{p^n}) = p$. 
\end{frame}

\begin{frame}
    \frametitle{Background: Field Extensions}
    Let $k, K$ be two fields. If there is an homomorphism $k \to K$, we can identify $k$ with a subfield of $K$. 
    In that case, $K$ is a \textbf{field extension} of $k$ which we denote by $k \subseteq K$. 

    Given any field $K$ we can construct the algebraic closure $\overbar{K}$ which is the smallest algebraically closed extension containing $K$.

    Some examples:
    \begin{itemize}
        \item $\QQ \subseteq \mathbb{R} \subseteq \mathbb{C}$
        \item $\FF_p \subseteq \FF_{p^2} \subseteq \FF_{p^3} \dots \subseteq \overline{\FF}_{p}$
    \end{itemize}
\end{frame}

\begin{frame}
    \frametitle{Weierstrass Form} 
    \begin{align*}
        a x^3 + b x^2 y + c x y^2 + d y^3 +& e x^2 + f x y + g y^2 + h x + i y + j = 0 \\
        &\big\downarrow \\
        y^2 + a xy + b y &= x^3 + c x^2 + d x + e \\
        &\bigg\downarrow \; \mathrm{char}(\FF) \neq 2, 3 \\
        y^2 = x^3 &+ a x + b
    \end{align*}
    Much easier to manage!
\end{frame}

\begin{frame}
    \frametitle{Elliptic Curves}
    \begin{definition}
        Let $\FF$ be a field. An elliptic curve $E$ defined over a field $\FF$ is given by 
        \[ E: y^2 = x^3 + a x + b \]
        for $a, b \in \FF$. For any extension $\FF \subseteq \mathbb{E}$ we define
        \[ E(\mathbb{E}) = \set{(x, y) \in \mathbb{E} \times \mathbb{E} \mid y^2 = x^3 + a x + b } \cup \set{ \infty } \]
    \end{definition}
    Mathematicians are often interested with $E(\QQ) \subseteq E(\mathbb{R}) \subseteq E(\mathbb{C})$ but we mostly consider the finite case.
\end{frame}

\begin{frame}
    \frametitle{Some elliptic curves}
    \framesubtitle{(In $E(\mathbb{R})$ since they look better...)}
    TODO: One singular with cusp, one node and three non singular
\end{frame}

\begin{frame}
    \frametitle{Fundamental Quantities}
    \begin{definition}
        Let $E: y^2 = x^3 + a x + b$ be an elliptic curve.

        The \textbf{discriminant} of $E$ is 
            \[ \Delta = -16 (4 a^3 + 27 b^2) \] 
        A curve is \textbf{singular} if $\Delta = 0$.

        If $E$ is non-singular the $j$\textbf{-invariant} of $E$ is
            \[ j(E) = -1728 \frac{(4 A)^3}{\Delta} \]
    \end{definition}
    \begin{theorem}
        Let $E, E'$ be two elliptic curves over $K$. Then $E \cong E'$ if and only if $j(E) = j(E')$. 
    \end{theorem}
\end{frame}

\begin{frame}
    \frametitle{The Group Law} 
    TODO: Picture group law
\end{frame}

\begin{frame}
    \frametitle{The Group Law: Formulae}
    Let $E: y^2 = x^3 + a x + b$ be an elliptic curve. Let $P_i = (x_i, y_i) \in E(K)$. Define 
    \[ -P_0 = (x_0, -y_0 )\]
    Now, for $P_1 + P_2$: 
    \begin{itemize}
        \item If $x_1 = x_2$ and $y_1 = -y_2$, then $P_1 + P_2 = \infty$
        \item If $P_1 = \infty$ then $P_1 + P_2 = P_2$, and viceversa.
        \item Let $x_3 = \lambda^2 - x_1 - x_2$, $y_3 = \lambda(x_1 - x_3) - y_1$ where $\lambda$ is defined as:
        \[ \lambda = \begin{cases}
            \frac{y_2 - y_1}{x_2 - x_1}, \; x_1 \neq x_2 \\
            \frac{3 x_1^2 + a}{2y_1},\; \text{otherwise}
        \end{cases} \]
    \end{itemize}
    
\end{frame}

\begin{frame}
    \frametitle{}

    

\end{frame}

\begin{frame}
    \frametitle{Resources}
    \begin{itemize}
        \item  J.H. Silverman, The Arithmetic of Elliptic Curves
        \item  J.H. Silverman, J.T. Tate, Rational Points on Elliptic Curves 
        \item  D.A. Cox, Primes of the form $x^2 + n y^2$
        \item  P. Aluffi, Algebra: Chapter 0
    \end{itemize}
\end{frame}


\end{document}